%
% recherche avec Atom dans le projet (*.tex)
% [\w-]+\*
%
% en bash
% (ne gère pas les mots contenant un tiret, exemple : "UTF-8")
% find documents -name '*.tex' | xargs sed -rn 's#^.*\b((\w|-)+)\*.*$#\1#p' | sort | uniq -i
%

\newglossaryentry{UTF-8}{
  name={UTF-8},
  description={Universal Character Set Transformation Format --- Codage de caractères informatiques compatible avec le standard Unicode.}
}

\newglossaryentry{AME}{
  name={AME},
  description={Département Aménagement, Mobilité et Environnement.}
}

\newglossaryentry{Atom}{
  name={Atom},
  description={Éditeur de texte développé par \gls{GitHub}.}
}

\newglossaryentry{CAF}{
  name={CAF},
  description={Caisse d'Allocations Familiales.}
}

\newglossaryentry{conda}{
  name={conda},
  description={Gestionnaire de paquets et d'environnements virtuels.}
}

\newglossaryentry{EPST}{
  name={EPST},
  description={Établissement Public à caractère Scientifique et Technologique.}
}

\newglossaryentry{git}{
  name={git},
  description={Logiciel de gestion de versions décentralisé.}
}

\newglossaryentry{GitHub}{
  name={GitHub},
  description={Service web d'hébergement et de gestion de développement de logiciels, qui utilise le logiciel de gestion de versions \gls{git}.}
}

\newglossaryentry{IDE}{
  name={IDE},
  description={Integrated Development Environment --- Environnement de développement.}
}

\newglossaryentry{IFSTTAR}{
  name={IFSTTAR},
  description={Institut Français des Sciences et Technologies des Transports, de l'Aménagement et des Réseaux.}
}

\newglossaryentry{IGN}{
  name={IGN},
  description={IInstitut géographique national.}
}

\newglossaryentry{INSEE}{
  name={INSEE},
  description={Institut National de la Statistique et des Études Économiques.}
}

\newglossaryentry{IP}{
  name={IP},
  description={Internet Protocol address --- Identifiant attribué à un appareil au sein d'un réseau informatique.}
}

\newglossaryentry{JetBrains}{
  name={JetBrains},
  description={Entreprise informatique éditrice, entre autres, de l'\gls{IDE} \bsc{PyCharm}.}
}

\newglossaryentry{JSON}{
  name={JSON},
  description={JavaScript Object Notation --- Notation objet empruntée au JavaScript.}
}

\newglossaryentry{LAMES}{
  name={LAMES},
  description={Laboratoire auscultation, modélisation, expérimentation des infrastructures de transport.}
}

\newglossaryentry{LaTeX}{
  name={LaTeX},
  description={Langage et système de composition de documents.}
}

\newglossaryentry{LTS}{
  name={LTS},
  description={Long-Term Support --- Support sur le long terme.}
}

\newglossaryentry{markdown}{
  name={markdown},
  description={Langage de balisage léger qui offre une syntaxe facile à lire et à écrire.}
}

\newglossaryentry{MIME}{
  name={MIME},
  description={Multipurpose Internet Mail Extensions --- Identifiant de format de données.}
}

\newglossaryentry{OS}{
  name={OS},
  description={Operating System --- Système d'exploitation.}
}

\newglossaryentry{pip}{
  name={pip},
  description={Pip Installs Packages --- Gestionnaire de paquets pour Python.}
}

\newglossaryentry{POSIX}{
  name={POSIX},
  description={Portable Operating System Interface for uniX --- Famille de normes techniques qui standardisent les interfaces de programmation des logiciels destinés à fonctionner sous un \gls{OS} \gls{UNIX}}
}

\newglossaryentry{PyCharm}{
  name={PyCharm},
  description={\gls{IDE} pour le Python développé par \bsc{JetBrains}.}
}

\newglossaryentry{RAM}{
  name={RAM},
  description={Random Access Memory --- Mémoire vive.}
}

\newglossaryentry{regex}{
  name={regex},
  description={REGular EXpressions --- Expressions "régulières".}
}

\newglossaryentry{SGBD}{
  name={SGBD},
  description={Système de Gestion de Base de Données.}
}

\newglossaryentry{SIG}{
  name={SIG},
  description={Système d'Information Géographique.}
}

\newglossaryentry{SRID}{
  name={SRID},
  description={Spatial Reference System Identifier --- Identifiant de Système de Référence.}
}

\newglossaryentry{SSH}{
  name={SSH},
  description={Secure Shell --- À la fois programme informatique et protocole de communication sécurisé.}
}

\newglossaryentry{SIG}{
  name={SIG},
  description={Système d'Information Géographique.}
}

\newglossaryentry{SublimeText}{
  name={SublimeText},
  description={Éditeur de texte développé par Jon \bsc{SKINNER}.}
}

\newglossaryentry{UNIX}{
  name={UNIX},
  description={Famille d'\gls{OS} dérivant d'un système d'exploitation multitâche et multi-utilisateur qui repose sur un interpréteur. On appelle "systèmes Unix" l'ensemble de ces systèmes dérivés.}
}

\newglossaryentry{URI}{
  name={URI},
  description={Uniform Resource Identifier --- Chaîne de caractères identifiant une ressource sur un réseau.}
}

\newglossaryentry{WKT}{
  name={WKT},
  description={Well-Known Text --- C'est un format standard en mode texte utilisé pour représenter des objets géométriques vectoriels issus de \gls{SIG}, mais aussi des informations s’y rattachant, tels les \gls{SRID}.}
}

\newglossaryentry{YAML}{
  name={YAML},
  description={\gls{YAML} Ain't Markup Language --- Format de données textuelles, aéré visuellement.}
}
