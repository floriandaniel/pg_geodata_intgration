\astitle{Résumé}

Le travail présenté dans ce rapport a trait à la mise en œuvre d'une application qui permet de créer et d'alimenter automatiquement une base de données géoréférencées. En effet, plusieurs problématiques traitées à l'\ifsttar nécessitent d'avoir à disposition des données, souvent volumineuses, relatives au territoire.  Ainsi, il est fréquemment nécessaire de se munir d'une base de données géoréférencées. Avant les travaux de ce stage, le processus de manipulation des données exploitées était rudimentaire. En effet, le stockage s'effectuait directement dans un système de fichiers. Également, les utilisateurs procédaient manuellement pour récupérer et transformer ces données, ce qui était laborieux et très coûteux en termes de temps.

Afin d'apporter une solution à ces problématiques, ce stage consiste à réaliser un ensemble d'outils auxquels seront déléguées ces tâches de récupération, de transformation et de stockage dans un serveur centralisé de données. Grâce à notre application, cette opération doit être aisée et reproductible~; c'est-à-dire pouvoir recréer et alimenter, à partir de zéro et sur commande, un serveur avec les données souhaitées. L'utilisateur final sera ainsi libéré des manipulations de pré-traitement, et pourra se concentrer davantage sur l'exploitation des données.

L'application a été développée en python. Elle consiste tout d'abord à lire un fichier récapitulant les adresses des sources de données voulues. En se basant sur ce fichier, les données sont récupérées, décompressées et explorées. Ensuite, une étape de pré-traitement est réalisée, qui consiste à remodeler ces données en fonction de paramètres fournis par l'utilisateur dans un fichier de configuration. Enfin, ces données sont importées dans le système de gestion de base de données PostgreSQL muni de l'extension géographique PostGIS. En outre, plusieurs scripts systèmes ont été développés pour installer et configurer automatiquement le serveur de données.

\astitle{Abstract}

The work exposed in this internship report concerns the production of an application. This application allows to create and fill automatically a database with geographic data. Several issues handled by \ifsttar needs having an access to datasets, usually huge ones, which are realtive to the territory. That is why it is necessary to possess a georeferenced database. Before this internship's work, the manipulating process of collected data was rudimentary. In fact, the storage was directly handled by the filesystem. Moreover, users were collecting and processing manually the data sets. This was laborious and really heavy, in terms of spent time.

In order to bring a solution that can answer these issues, this internship consists in the production of a toolbox. The retrieving, transforming, and storage tasks in a centralized data platform will be delegated to these tools. Thanks to our application, this operation must be simple and reproducible. More precisely, it means to have the aptitude of re-creating and fill, from scratch and on demand, a data platform containing all the required data. Therefore, the final user will be released from pre-processing constraints so he can have a better focus on the data analysis.

The application was developed in Python. First of all, the application consists in reading the data sets sources specifications, which are compiled in a configuration file. From this point, the data is retrieved, extracted and explored. Then, a pre-processing step is applied. This step allows to remodel the data, taking into consideration the parameters given by the user from a configuration file. At last, these data are imported in a PostgreSQL's database management system, equipped with the PostGIS extension. In addition, several shell scripts have been developed. They are in charge of installing and configuring automatically the data server.

\clearpage
